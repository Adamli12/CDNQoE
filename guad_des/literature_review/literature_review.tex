\documentclass{ctexart}
\CTEXoptions[today=old]
\renewcommand\refname{Reference}
\usepackage[colorlinks,linkcolor=red, anchorcolor=blue, citecolor=green]{hyperref}
\title{QoE driven CDN resource allocation approach: literature review}
\author{Hanyu Li}
\begin{document}
\maketitle
\section{The method that is being used now by industry}
The most commonly used method is randomly assign new users to currently available CDN(Content Delivery Networks), usually according to the hashed id of users. Apparently, this kind of round robin approach gives rise to a set of problems. For example, the dynamic CDN status is neglected. Furthermore, the QoE(Quality of Experience) of users are affected by various factors which may contain factors other than QoS(Quality of Service).
\section{State of art works}
\subsection{Prediction based}
\subsubsection{CFA}
\par It aims to capture the complex relationships between session features and vedio quality
\par its pros and cons
\subsection{E2 based}
\subsection{user heterogeneity}
\section{User heterogeneity: a new way to improve QoE}
\section{How we mitigate those drawbacks}


\nocite{*}
\bibliographystyle{unsrt}
\bibliography{lib}
\end{document}